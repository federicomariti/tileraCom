\section{Descrizione dell'applicazione}
\`E stata realizzata una applicazione sul calcolo del prodotto matrice per vettore. La computazione considerata \`e su stream, gli elementi dello stream sono le matrici su cui effettuare il calcolo, mentre il vettore rimane costante per tutta l'esecuzione dell'applicazione. 
L'implementazione scelta fa uso del paradigma Data Parallel, in particolare la forma \emph{Map}, in quanto ogni processo worker andr\`a ad eseguire un calcolo che \`e indipendente dall'attivit\`a degli altri processi. Si \`e scelto il partizionamento delle matrici per riga. 
Per realizzare le comunicazioni vengono usati i canali del supporto fornito, quindi si ha lo scambio dei puntatori alle strutture dati condivise. L'applicazione risulta seguire lo schema \emph{multicast}-\emph{compute}-\emph{gather} in quanto per ogni matrice dello stream si eseguono le seguenti tre fasi: 
\begin{compactenum}[\itshape a~\upshape)] 
\item la distribuzione del riferimento alla matrice corrente tra i processi worker, \`e compito di ogni worker effettuare il calcolo nella propria partizione della matrice, 
\item il calcolo dei risultati parziali in ciascun worker, 
\item la raccolta di tutti i risultati parziali dell'elemento.
\end{compactenum}
La comunicazione collettiva \emph{multicast} \`e implementata con una struttra ad albero binario mappato nell'insieme dei processi worker, le comunicazioni che costituiscono l'albero sono implementate per mezzo dei canali simmetrici offerti dal supporto; l'altra comunicazione collettiva, la \emph{gather}, viene implementata per mezzo del canale asimmetrico in ingresso fornito dal supporto. \\
L'applicazione \`e fittizia, nel senso che non esistono dispositivi che generino e collezionino lo stream, per questo motivo l'applicazione \`e costituita, oltre che dai processi che realizzano la \emph{map}, da altri due processi che, rispettivamente, generano e collezionano gli elementi dello stream. 
Tali due processi comunicano con il sottositema dei workers (\subsystem) per mezzo dei canali del supporto: il processo generatore \`e collegato tramite un canale simmetrico al worker radice dell'albero multicast, ogni processo worker \`e collegato al processo collettore per mezzo di un canale asimmetrico.

Riassumendo, ogni processo worker fa uso di 4 canali di comunicazione: 
\begin{asparaitem}
\item tre sono simmetrici e trasportano gli elementi dello stream, uno dei quali \`e in ingresso dal processo padre nell'albero della multicast, gli altri due sono in uscita verso i processi radice dei due sottalberi della multicast, 
\item il quarto canale \`e asimmetrico in ingresso ed \`e usato in scrittura, per la comunicazione del risultato parziale al processo collezionatore.
\end{asparaitem}
\`E quindi possibile l'uso del supporto alle comunicazioni che fa uso della UDN in quanto, per ogni processo, il numero di canali non supera il numero di code hardware.
